\documentclass[a4paper,twocolumn]{article}

% \usepackage{amsmath,amssymb,color,graphicx,hyperref,listings}
% \usepackage{mathtools,mdwlist}

\usepackage{url}
\usepackage[margin=2cm]{geometry}

\title{
  koneko: a mew (dys)functional programming language stacked to be
  pointlessly concatenative as well as argumentatively lambdalicious
}
\author{Felix C. Stegerman $<$flx@obfusk.net$>$}
\date{2019-10-07}

\begin{document}

\maketitle

\abstract

Koneko is a simple concatenative stack-based programming language with
lisp influences.  It is intended to combine the elegance of the
(point-free) "concatenation is composition" model with the elegance of
lisp-like languages (esp. anonymous functions with named arguments).

We describe the syntax and semantics, evaluate the prototype
interpreter we implemented (in Haskell), identify the advantages and
disadvantages of the language, and compare koneko to other languages
and language families.

\section{Introduction}

% ### Properties
%
% * concatenative
%   - point-free
%   - juxtaposition of expressions denotes function composition
% * stack-oriented
%   - postfix (reverse polish) notation
%   - functions consume arguments from the stack
%   - functions produce return values on the stack
% * lisp-like
%   - homoiconic
%   - blocks/lambdas (anonymous functions)
%   - named arguments/points (lexically scoped)
% * functional
%   - only immutable data structures
%   - does have side effects (I/O)
%   - strict evaluation
% * dynamically, strongly typed

% unieke/onbekende combinatie van features
% weinig gebruikt evaluatie model (modulo PostScript)
% voor- en nadelen van deze combinatie onbekend

% what comes next

\section{The koneko Language}

\subsection*{Syntax}

\subsection*{Semantics}

\subsection*{Future Work}

\section{Examples}

\section{Prototype Interpreter}

\subsection*{Design Choices}

\subsection*{Implementation Choices}

\subsection*{Refection}

\section{Related Work}

% --> https://concatenative.org/wiki/view/Concatenative%20language/Publications

\subsection*{Comparisons to Other Languages}

\section{Conclusions}

\section{Links}

The source code of the interpreter is available online at
\url{https://github.com/obfusk/koneko}.

% ... \cite{bird&wadler} ...

% \bibliographystyle{alpha}
% \bibliography{references}

% CC-BY-SA https://creativecommons.org/licenses/by-sa/4.0/

\end{document}
